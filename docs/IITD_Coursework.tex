\documentclass[a4paper]{article}
\usepackage[margin=1cm]{geometry}
\usepackage{subcaption}
\usepackage{graphicx} % Required for inserting images
\usepackage{wrapfig}
\usepackage{xcolor}
\title{IITD Coursework}
\author{Akshat Jha - abowtiekindofguy}
\date{Updated May 2024}
\begin{document}
\maketitle
\section*{Diwali Semester 2023-24}
\begin{itemize}
    \item \textbf{COL106 Data Structures and Algorithms} {\color{red}A}
    \\
    Introduction to object-oriented programming through stacks queues and linked lists. Dictionaries; skip-lists, hashing, analysis of collision resolution techniques. Trees, traversals, binary search trees, optimal and average BSTs. Balanced BST: AVL Trees, 2-4 trees, red-black trees, B-trees. Tries and suffix trees. Priority queues and binary heaps. Sorting: merge, quick, radix, selection and heap sort, Graphs: Breadth first search and connected components. Depth first search in directed and undirected graphs. Dijkstra’s algorithm, directed acyclic graphs and topological sort. Spanning Tree Algorithms. Binomial and Fibonacci Heaps. Search Techniques. Probabilistic Algorithms and Bloom Filters (Didn't attend this part)
    \\
    \textit{Textbook}: Goodrich and Tamassia (C++ version)
    \\
    \textit{Remarks}: A bit poorly managed. Easy exams. Very involved assignments that taught me OOPS and memory management (more than algorithms)
    \item \textbf{COL215 Digital Logic and System Design} {\color{red}A}
    \\
    Introduction to Digital Logic and Basics of Transistors. Combinatorial Logic Design. Sequential Circuit Design. Counters, Memories and Registers. 
    \\
    \textit{Textbook}: None
    \\
    \textit{Remarks}: Highly dependent on instructor. 2-3 difficult hardware assignments with limited resources. Very subjective
    \item \textbf{COL202 Discrete Mathematics} {\color{red}A}
    \\
    Propositional logic, Predicate Calculus and Quantifiers; Proof Methods; Sets, functions, relations, Cardinality, Infinity and Diagonalization; Induction and Recursion; Basic Counting - Pigeon hole principle; Advanced Counting - recurrence relations, generating functions, inclusion- exclusion; Probability - sample space, conditional probability, expectation, linearity of expectation, variance, Markov, Chebychev, probabilistic methods; Graph Theory - Eulerian, Hamiltonian and planar graphs, edge and vertex coloring. Basically all parts of LLM Book except Number Theory.
    \\
    \textit{Textbook}: LLM Book
    \\
    \textit{Remarks}: Don't commit silly mistakes and solve LLM Book's problems. Previous offerings' tutorials and Rohit Vaish's problems are another excellent source.
    \item \textbf{MTL106 Introduction to Probability and Stochastic Processes} {\color{red}A}
    \\
    Axioms of probability, Probability space, Conditional probability, Independence, Bayes’ rule, Random variable, Some common discrete and continuous distributions, Distribution of Functions of Random Variable, Moments, Generating functions, Two and higher dimensional distributions, Functions of random variables, order statistics, Conditional distributions, Covariance, Correlation coefficient, conditional expectation, Modes of convergences, Laws of large numbers, Central limit theorem, Definition of Stochastic process, Classification and properties of stochastic processes, Simple Markovian stochastic processes, Gaussian processes, Stationary processes, Discrete and continuous time Markov chains, Classification of states, Limiting distribution, Birth and death process, Poisson process, Steady state and transient distributions, Simple Markovian queuing models (M/M/1, M/M/1/N, M/M/c/N, M/M/N/N, M/M/inf).
    \\
    \textit{Textbook}: Introduction to Probability and Stochastic Processes with Applications by Liliana Blanco Castañeda, Viswanathan Arunachalam and Delvamuthu Dharmaraja
    \\
    \textit{Remarks}: Easy course if you do the book and attend Dharma's lectures. Very high cutoffs in my offering due to an easy Major
    \item \textbf{CVL100 Environmental Science} {\color{red}A-}
    \\
    Random Assortment of slides on Water and Air pollution
    \\
    \textit{Textbook}: None
    \\
    \textit{Remarks}: Don't commit silly mistakes! Very instructor dependent. Fairly graded. Meaningless course
    
\end{itemize}
\newpage
\section*{Holi Semester 2023-24}
\begin{itemize}

\item \textbf{COL216 Computer Architecture} {\color{red}A}
    \\
    History of computers, Boolean logic and number systems, Assembly language programming, ARM assembly language, Computer arithmetic, Design of a basic processor, Microprogramming, Pipelining, Memory system, Virtual memory, I/o protocols and devices, Multiprocessors
    \\
    \textit{Textbook}: Computer Organisation and Design (RISC V Edition)
    \\
    \textit{Remarks}: The book is everything. A bit weird instruction in my semester. But very interesting and easy if you understand the basics
\item \textbf{COL226 Programming Languages} {\color{red}A}
    \\
    Intro to Functional Programming. Sets and Relations. Big Step Semantics. Operational and Denotational Semantics. Regular Expressions. Lex and Yacc. Imperative Languages. Hoare Logic. 
     \\
    \textit{Textbook}: Programming Language Pragmatics by Michael Scott and Sanjiva Prasad's Notes
    \\
    \textit{Remarks}: COL226 Notes are a excellent introduction to Rigorous PL. Although a very oddly designed course
\item \textbf{COL726 Numerical Algorithms} {\color{red}A}
    \\
    Number representation, fundamentals of error analysis, conditioning, stability, polynomials and root finding, interpolation, singular value decomposition and its applications, QR factorization, condition number, least squares and regression, Gaussian elimination, eigenvalue computations and applications, iterative methods, linear programming, elements of convex optimization including steepest descent, conjugate gradient, Newton’s method.
     \\
    \textit{Textbook}: Trefethen and Bau's Numerical Linear Algebra and Heath's Introduction to Scientific Computing
    \\
    \textit{Remarks}: Very well taught lectures by Prof. Amit Kumar. Programming Assignments should be more involved and maybe a project would be nice. The homeworks were challenging and exams were perfectly set.
    \item \textbf{COP290 Discrete Mathematics} {\color{red}A}
    \\
    Created a Flask Website and a Flutter Android App for the course projects
    \\
    \textit{Textbook}: None
    \\
    \textit{Remarks}: A bit demanding but lenient grading
    \item \textbf{HUL212 Microeconomics} {\color{red}A}
    \\
    Micro versus macroeconomics. Theory of consumer behavior and demand. Consumer preferences. Indifference curve. Consumer equilibrium. Demand function. Income and substitution effects. The Slutsky equation. Market demand. Elasticities. Average and marginal revenue. Revealed preference theory of firm. Production functions. Law of variable proportions. Laws of return to scale. Isoquants. Input substitution. Equilibrium of the firm. Expansion path. Cost function. Theory of costs. Short Run and Long run costs. Shape of LAC. Economies and diseconomies of scale. Market equilibrium under perfect competition. Equilibrium under alternative forms of market. Monopoly: pure and discriminating. Monopolistic competition. oligopoly. Marriage Markets. Exchange Economies. Introduction to Game Theory.
     \\
    \textit{Textbook}: Hal Varian's Microeconomics
    \\
    \textit{Remarks}: Must do if Rohit Kumar is teaching. Extremely good instruction quality in lectures.
\item \textbf{ELL205 Signals and Systems} {\color{red}A}
    \\
    Classifications of signals and systems, Dynamic representation of LTI systems (discrete and continuous-time systems), Fourier analysis of continuous-time and systems, Fourier analysis of discrete-time signals and systems, Nyquist sampling theorem, Laplace transform, The z-transform, Introduction to probability, random variables and stochastic processes.
     \\
    \textit{Textbook}: Oppenheim's Signals and Systems
    \\
    \textit{Remarks}: The even semester offering for CSE students is really easy. Very easy to get through with just basic idea about the transforms and techniques
    \item \textbf{SBL100 Introduction to Biology for Engineers} {\color{red}A}
    \\
    Random Assortment of Slides on Cytology, Immunology and Ecology. 
     \\
    \textit{Textbook}: None
    \\
    \textit{Remarks}: CVL100 but 4-credit. Got an A by luck. Don't commit silly mistakes in the exams and get full in labs
    
\end{itemize}
\end{document}

